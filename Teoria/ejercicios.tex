\documentclass[11pt,a4paper]{article}
\usepackage[utf8]{inputenc}
\usepackage[spanish]{babel}	%Idioma
\usepackage{amsmath}
\usepackage{amsfonts}
\usepackage{amssymb}
\usepackage{graphicx} 	%Añadir imágenes
\usepackage{geometry}	%Ajustar márgenes
\usepackage[export]{adjustbox}[2011/08/13]
\usepackage{float}
\restylefloat{table}
\usepackage[hidelinks]{hyperref}
\usepackage{titling}
\graphicspath{{/home/nazaret/Escritorio/LaTEX}}
%\usepackage{minted}
\usepackage{multirow}
\usepackage{caption}
\usepackage{multicol}
\usepackage[shortlabels]{enumitem}
\usepackage{array}
\selectlanguage{spanish}

%Opciones de encabezado y pie de página:
\usepackage{fancyhdr}
\pagestyle{fancy}
\lhead{Nazaret Román Guerrero}
\rhead{Procesamiento Digital de Señales}
\lfoot{Grado en Ingeniería Informática}
\cfoot{}
\rfoot{\thepage}
\renewcommand{\headrulewidth}{0.4pt}
\renewcommand{\footrulewidth}{0.4pt}

%Opciones de fuente:
\usepackage[utf8]{inputenc}
\usepackage[default]{sourcesanspro}
\usepackage{sourcecodepro}
\usepackage[T1]{fontenc}

\setlength{\parindent}{15pt}
\setlength{\headheight}{15pt}
\setlength{\voffset}{10mm}

% Custom colors
\usepackage{color}
\definecolor{deepblue}{rgb}{0,0,0.5}
\definecolor{deepred}{rgb}{0.6,0,0}
\definecolor{deepgreen}{rgb}{0,0.5,0}

\usepackage{listings}
\usepackage{color}
\usepackage{graphicx}

\definecolor{dkgreen}{rgb}{0,0.6,0}
\definecolor{gray}{rgb}{0.5,0.5,0.5}
\definecolor{mauve}{rgb}{0.58,0,0.82}

\lstset{frame=tb,
  language=Matlab,
  aboveskip=3mm,
  belowskip=3mm,
  showstringspaces=false,
  columns=flexible,
  basicstyle={\small\ttfamily},
  numbers=left,
  numberstyle=\tiny\color{gray},
  keywordstyle=\color{blue},
  commentstyle=\color{dkgreen},
  stringstyle=\color{mauve},
  breaklines=true,
  breakatwhitespace=true,
  tabsize=4
}

\begin{document}
\begin{titlepage}

\begin{minipage}{\textwidth}

\centering
\includegraphics[width=0.55\textwidth]{img/logo.png}\\

\textsc{\Large Procesamiento Digital de Señales\\[0.2cm]}
\textsc{GRADO EN INGENIERÍA INFORMÁTICA}\\[1cm]

{\Huge\bfseries Relación de ejercicios\\}
\noindent\rule[-1ex]{\textwidth}{3pt}\\[3.5ex]
{\large\bfseries Ejercicios de teoría}
\end{minipage}

\vspace{1.5cm}
\begin{minipage}{\textwidth}
\centering

\textbf{Autora}\\ {Nazaret Román Guerrero}\\[2.5ex]
\includegraphics[width=0.3\textwidth]{img/etsiit.jpeg}\\[0.1cm]
\vspace{1cm}
\textsc{Escuela Técnica Superior de Ingenierías Informática y de Telecomunicación}\\
\vspace{1cm}
\textsc{Curso 2018-2019}
\end{minipage}
\end{titlepage}

\pagenumbering{gobble}
\pagenumbering{arabic}
\tableofcontents
\thispagestyle{empty}

\newpage

\section{Ejercicio 14 (Problemas de clase)}

Un sistema LTI se encuentra caracterizado por la siguiente ecuación recursiva

\[y(n)=0.5y(n-1)+x(n)\]

Calcule la salida $y(n)$ de forma recursiva para $x(n)=u(n+1)-u(n-1)$ e $y(0)=1$.\\

Para calcular la salida voy a utilizar el polinomio para resolver recurrencias lineales. Para ello vamos a sacar las raíces del polinomio:

	\begin{gather*}
	P(y) = y - 0.5; \\
	y = 0.5
	\end{gather*}
	
	Por tanto sabemos que $y(n)=0.5^n$ es solución de la recurrencia. Vamos a comprobarlo desarrollando la recurrencia original para saber si estamos en lo correcto:

	\begin{gather*}
	n=0 \longrightarrow x(0) = 1-0 = 1 \longrightarrow y(0)=1 \\
	n=1 \longrightarrow x(1) = 1-1 = 0 \longrightarrow y(1)=0.5\cdot 1+0=0.5\\
	n=2 \longrightarrow x(2) = 1-1 = 0 \longrightarrow y(2)=0.5\cdot 0.5 + 0=0.25\\
	n=3 \longrightarrow x(3) = 1-1 = 0 \longrightarrow y(3)=0.5\cdot 0.25 + 0=0.125
	\end{gather*}
	
Como podemos ver en el desarrollo anterior, efectivamente la sucesión original es la que se corresponde con $y(n)=0.5^n$



\end{document}